\documentclass{article}
\usepackage[osf,p]{libertinus}
\usepackage{microtype}
\usepackage[pdfusetitle,hidelinks]{hyperref}
\usepackage[series={},nocritical,noend,nofamiliar,noledgroup]{reledmac}
\usepackage{reledpar}

\usepackage{graphicx}
\usepackage{polyglossia}
\setmainlanguage{english}
\setotherlanguage{hebrew}
\gappto\captionshebrew{\renewcommand\chaptername{קאַפּיטל}}
\usepackage{metalogo}

\linenumincrement*{1}
\firstlinenum*{1}
\setlength{\Lcolwidth}{0.44\textwidth}
\setlength{\Rcolwidth}{0.44\textwidth}

\begin{document}

\title{The Devout Cat \\ קאַץ פֿרומע די}
\author{YL Peretz\\Translated by Ilan Pillemer}
\date{\today}

\maketitle
\pstartnumfalse
\begin{pairs}

\begin{Rightside}
\begin{RTL}
\begin{hebrew}
\beginnumbering


\pstart
דרײַ זינגלפֿײגעלעך זײַנען געװען אין אײן שטוב, און אַלע דרײַ, אײנס נאָכן צװײטן - האָט די קאַץ צורעכט געמאַכט...
\pend
\pstart
עס איז קײן אײנפֿאַכע קאַץ נישט געװען, נאָר אַן אמת, אמת פֿרומע נשמה.
\pend
\pstart
עס איז אַ פֿרומע קאַץ, אַ טבֿילה-קאַץ געװען! צען מאָל אין טאָג האָט זי זיך געװאַשן, און געגעסן האָט זי שטיל, ערגעץ אין אַ זײַט, אין אַ װינקעלע...
\pend
\endnumbering
\end{hebrew}
\end{RTL}
\end{Rightside}




\begin{Leftside}
\begin{english}
\beginnumbering
\pstart
Three dear songbirds had been in one house, and all three, one after the other - had been finished off by the cat...
\pend
\pstart
It was no ordinary cat, but a truly, truly devout soul. 
\pend
\pstart
It was a devout cat, a tevilah\textsuperscript{1}-cat!
Ten times a day, she washed herself, and then she ate quietly, once in a while, in a little corner.
\pend
 \footnote{Tevilah refers to both the traditional religiously required bathing in flowing water or a Mikvah, as well as the traditional and also religiously required cleaning of the hands through the pouring of water before eating a meal.} 
\endnumbering
\end{english}
\end{Leftside}

\end{pairs}
\Columns


\end{document}
