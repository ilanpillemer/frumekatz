\documentclass{book}
\usepackage[osf,p]{libertinus}
\usepackage{microtype}
\usepackage[pdfusetitle,hidelinks]{hyperref}
\usepackage[series={},nocritical,noend,nofamiliar,noledgroup]{reledmac}
\usepackage{reledpar}

\usepackage{graphicx}
\usepackage{polyglossia}
\setmainlanguage{english}
\setotherlanguage{hebrew}
\gappto\captionshebrew{\renewcommand\chaptername{קאַפּיטל}}
\usepackage{metalogo}

\linenumincrement*{1}
\firstlinenum*{1}
\setlength{\Lcolwidth}{0.44\textwidth}
\setlength{\Rcolwidth}{0.44\textwidth}

\begin{document}

\title{The Frum Cat}
\author{YL Peretz\\Translated by Ilan Pillemer}
\date{}

\maketitle
\begin{pairs}

\begin{Rightside}
\begin{RTL}
\begin{hebrew}
\beginnumbering
\pstart
די פֿרומע קאַץ
\newline
\pend
\pstart
דרײַ זינגלפֿײגעלעך זײַנען געװען אין אײן שטוב, און אַלע דרײַ, אײנס נאָכן צװײטן - האָט די קאַץ צורעכט געמאַכט...
\pend
\pstart
עס איז קײן אײנפֿאַכע קאַץ נישט געװען, נאָר אַן אמת, אמת פֿרומע נשמה.

\pend
\endnumbering
\end{hebrew}
\end{RTL}
\end{Rightside}




\begin{Leftside}
\begin{english}
\beginnumbering
\pstart
The Devout Cat
\newline
\pend
\pstart
Three songbirds had been in one house, and all three, one after the other - had been finished off by the cat...
\pend
\pstart
This was no ordinary cat, but a truly, truly devout soul.
\pend
\endnumbering
\end{english}
\end{Leftside}

\end{pairs}
\Columns
\end{document}
